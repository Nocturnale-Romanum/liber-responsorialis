% !TEX TS-program = lualatex
% !TEX encoding = UTF-8

\documentclass[liber-responsorialis.tex]{subfiles}

\ifcsname preamble@file\endcsname
  \setcounter{page}{\getpagerefnumber{M-lr31_commune_martyrum}}
\fi

\begin{document}
\feast{UMEX}{Commune unius Martyris\\extra Tempus Paschale}
	{Commune Sanctorum}{Commune unius Martyris extra T. P.}{2}{}{}{}{}{}{}
\addcontentsline{toc}{section}{Commune Martyrum}

\nocturn{1}
\gscore{UMEXN1R1}{R}{1}{Iste Sanctus pro lege}
\gscore{UMEXN1R2}{R}{2}{Justus germinabit sicut lilium}
\gscore{UMEXN1R3}{R}{3}{Iste cognovit justitiam}
\nocturn{2}
\gscore{UMEXN2R1}{R}{4}{Honestum fecit illum}
\gscore{UMEXN2R2}{R}{5}{Desiderium animae ejus}
\gscore{UMEXN2R3}{R}{6}{Stolam jucunditatis induit}
\nocturn{3}
\gscore{UMEXN3R1}{R}{7}{Corona aurea super caput}
\gscore{UMEXN3R2a}{R}{8}{Hic est vere Martyr}
\rubric{Sequens Responsorium dicitur, loco præcedentis, in Officio unius Martyris, qui non effuso sanguine occubuerit.}
\gscore{UMEXN3R2b}{R}{8}{Domine praevenisti eloria}
\tedeumrubric

\feast{PMEX}{Commune plurimorum Martyrum\\extra Tempus Paschale}
	{Commune Sanctorum}{Commune plurimorum Martyrum extra T. P.}{2}{}{}{}{}{}{}
\nocturn{1}
\gscore{PMEXN1R1}{R}{1}{Absterget Deus}
\gscore{PMEXN1R2}{R}{2}{Viri sancti gloriosum sanguinem}
\gscore{PMEXN1R3}{R}{3}{Tradiderunt corpora sua}
\nocturn{2}
\gscore{PMEXN2R1}{R}{4}{Sancti tui Domine mirabile}
\gscore{PMEXN2R2}{R}{5}{Verba carnificum}
\gscore{PMEXN2R3}{R}{6}{Tamquam aurum in fornace}
\nocturn{3}
\gscore{PMEXN3R1}{R}{7}{Propter testamentum Domini}
\gscore{PMEXN3R2a}{R}{8}{Sancti mei qui in isto}
\rubric{Sequens Responsorium dicitur, loco præcedentis, in Officio plurimorum Martyrum Fratrum, etiam si ii recolantur cum Sociis; dummodo tamen Fratres vel numero sint plures, vel, si pares, primo loco nominentur.}
\gscore{PMEXN3R2b}{R}{8}{Haec est vera Fraternitas}

\feast{MRTP}{Commune unius aut plurimorum Martyrum\\Tempore Paschali}
	{Commune Sanctorum}{Commune unius aut plurimorum Martyrum T. P.}{2}{}{}{}{}{}{}

\rubric{Responsoria e Communi Apostolorum et Evangelistorum Tempore Paschali, pag. \pageref{M-APTPN1R1}, præter sequentes:}
\gscore{MRTPN2R2}{R}{5}{In servis suis}
\gscore{MRTPN2R3}{R}{6}{Filiae Jerusalem venite}

\end{document}