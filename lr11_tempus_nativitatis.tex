% !TEX TS-program = lualatex
% !TEX encoding = UTF-8

\documentclass[liber-responsorialis_temporale.tex]{subfiles}

\ifcsname preamble@file\endcsname
  \setcounter{page}{\getpagerefnumber{M-lr11_tempus_nativitatis}}
\fi

\begin{document}
%% this subfile begins on a LEFT, EVEN page.
\feast{1225}{In Nativitate\\Domini nostri Jesu Christi}
	{Proprium de Tempore}{In Nativitate Domini}{1}{25 Decembris}
	{Duplex I. classis}{I. classis}{Jesu Christi, Domini nostri!Nativitas}
	{}
	{}
\addcontentsline{toc}{section}{Tempus Nativitatis}
\sixlinesvspace
\nocturn{1}
\gscore{1225N1R1}{R}{1}{Hodie nobis caelorum Rex}
\sixlinesvspace
\sixlinesvspace
\pagebreak
\gscore{1225N1R2}{R}{2}{Hodie nobis de caelo pax}
\sixlinesvspace
\sixlinesvspace
\pagebreak
\gscore{1225N1R3}{R}{3}{Quem vidistis pastores}
\pagebreak
\nocturn{2}
\gscore{1225N2R1}{R}{4}{O magnum mysterium}
\gscore{1225N2R2}{R}{5}{Beata Dei Genitrix}
\pagebreak
\gscore{1225N2R3}{R}{6}{Sancta et immaculata}
\threeplustitlevspace
\nocturn{3}
\gscore{1225N3R1}{R}{7}{Beata viscera}
\sixlinesvspace
\gscore{1225N3R2}{R}{8}{Verbum caro factum est}

\pagebreak

\sixlinesvspace
\feast{N1F1}{Dominica infra Octavam Nativitatis}
	{Proprium de Tempore}{Dominica infra Octavam Nativitatis}{2}{}
	{Semiduplex Dominica minor}{II. classis}{Jesu Christi, Domini nostri!Nativitas, dominica infra octavam}
	{Omnia ut infra.}
	{Responsoria 1. et 3. infra.}
\nocturn{1}
\respref{1}{1225N1R2}{}
\resprefsinegp{2}{1225N1R3}
\resprefcumgp{3}{1225N2R1}
\nocturn{2}
\respref{4}{1225N2R2}{}
\resprefsinegp{5}{1225N1R3}
\resprefcumgp{6}{1225N3R1}
\nocturn{3}
\resprefsinegp{7}{1225N3R2}
\gscore{N1F1N3R2}{R}{8}{O Regem caeli}
\sixlinesvspace
\pagebreak

\feast{1230}{Die VI infra Octavam Nativitatis}
	{Proprium de Tempore}{Infra Octavam Nativitatis}{3}{30 Decembris}
	{Semiduplex}{II. classis}{}
	{Omnia ut in Dominica infra Octavam.}
	{Responsoria 1. et 2. de Dominica infra Octavam.}

\pagebreak

\feast{0101}{In Circumcisione Domini\\et Octava Nativitatis}
	{Proprium de Tempore}{In Circumcisione Domini}{2}{1 Januarii}
	{Duplex II. classis}{II. classis}{Jesu Christi, Domini nostri!Circumcisio}
	{}
	{}
\sixlinesvspace
\nocturn{1}
\gscore{0101N1R1}{R}{1}{Ecce Agnus Dei}
\sixlinesvspace
\gscore{0101N1R2}{R}{2}{Dies sanctificatus}
\sixlinesvspace
\sixlinesvspace
\gscore{0101N1R3}{R}{3}{Benedictus qui venit}

\nocturn{2}
\gscore{CBMVN1R2}{R}{4}{Congratulamini... parvula}
\gscore{0101N2R2}{R}{5}{Confirmatum est cor}
\pagebreak
\threelinesvspace
\index[R]{Benedicta et venerabilis@2 & Benedicta et venerabilis}
\gscore{0101N2R3}{R}{6}{Beata et venerabilis}
\newpage
\sixlinesvspace
\nocturn{3}
\resprefsinegp{7}{1225N2R3}{}
\gscore{0101N3R2}{R}{8}{Nesciens mater}
\sixlinesvspace
\sixlinesvspace
\sixlinesvspace
\pagebreak

\feast{N2}{In Festo Sanctissimi Nominis Jesu}
	{Proprium de Tempore}{Sanctissimi Nominis Jesu}{2}{Dominica inter Circumcisionem et Epiphaniam}
	{Duplex II. classis}{II. classis}{Jesu Christi, Domini nostri!Nomen}
	{}
	{}
\nocturn{1}
\gscore{N2N1R1}{R}{1}{Ecce concipies}
\gscore{N2N1R2}{R}{2}{Benedictum est nomen}
\pagebreak
\gscore{N2N1R3}{R}{3}{Laudabo nomen}
\sixlinesvspace
\sixlinesvspace
\pagebreak
\nocturn{2}
\gscore{N2N2R1}{R}{4}{Sperent in te}
\fiveplustitlevspace
\gscore{N2N2R2}{R}{5}{Confiteamur nomini}
\sixlinesvspace
\gscore{N2N2R3}{R}{6}{Laetentur omnes}
\sixlinesvspace
\pagebreak
\nocturn{3}
\gscore{N2N3R1}{R}{7}{Tribulationem et dolorem}
\sixlinesvspace
\fiveplustitlevspace
\pagebreak
\gscore{N2N3R2}{R}{8}{Exspectabo nomen tuum}
\pagebreak

\feast{0102}{In Octava S. Stephani Protomartyris}
	{Proprium de Tempore}{In Octava S. Stephani}{2}{2 Januarii}
	{Simplex}{(Omittitur)}{Stephani!Octava}
	{}
	{}
\respref{1}{0101N1R1}{}
\resprefcumgp{2}{0101N1R2}

\feast{0103}{In Octava S. Joannis\\Apostoli et Evangelistæ}
	{Proprium de Tempore}{In Octava S. Joannis}{2}{3 Januarii}
	{Simplex}{(Omittitur)}{Joannis Evangelistæ!Octava}
	{}
	{}
\respref{1}{CBMVN1R2}{}
\resprefcumgp{2}{0101N2R2}

\feast{0104}{In Octava Ss. Innocentium Martyrum}
	{Proprium de Tempore}{In Octava SS. Innocentium}{2}{4 Januarii}
	{Simplex}{(Omittitur)}{Innocentium!Octava}
	{}
	{}
\resprefsinegp{1}{1225N2R3}{}
\respref{2}{0101N3R2}{}

\feast{0105}{In Vigilia Epiphaniæ}
	{Proprium de Tempore}{In Vigilia Epiphaniæ}{2}{5 Januarii}
	{Semiduplex II. classis}{(Omittitur)}{Jesu Christi, Domini nostri!Epiphania, vigilia}
	{}
	{}
\rubric{Omnia ut in Octava Nativitatis.}
\pagebreak

\sixlinesvspace
\feast{0106}{In Epiphania\\Domini nostri Jesu Christi}
	{Proprium de Tempore}{In Epiphania Domini}{1}{6 Januarii}
	{Duplex I. classis}{I. classis}{Jesu Christi, Domini nostri!Epiphania}
	{}
	{}
\sixlinesvspace
\nocturn{1}
\gscore{0106N1R1}{R}{1}{Hodie in Jordane}
\sixlinesvspace
\gscore{0106N1R2}{R}{2}{In columbae specie}
\sixlinesvspace
\sixlinesvspace
\pagebreak
\gscore{0106N1R3}{R}{3}{Reges Tharsis}
\sixlinesvspace
\pagebreak
\nocturn{2}
\gscore{0106N2R1}{R}{4}{Illuminare}
\pagebreak
\gscore{0106N2R2}{R}{5}{Omnes de Saba}
\sixlinesvspace
\pagebreak
\gscore{0106N2R3}{R}{6}{Magi veniunt}
\pagebreak
\nocturn{3}
\gscore{0106N3R1}{R}{7}{Stella quam viderant}
\pagebreak
\gscore{0106N3R2}{R}{8}{Videntes stellam Magi}

\feast{E1F1}{In Festo sanctæ Familiæ\\Jesu, Mariæ, Joseph}
	{Proprium de Tempore}{Sanctæ Familiæ}{2}{Dominica infra Octavam Epiphaniæ}
	{Duplex majus}{II. classis}{Jesu Christi, Domini nostri!Familia Sancta}
	{}
	{}
\nocturn{1}
\gscore{E1N1R1}{R}{1}{Deus noster in terris}
\gscore{E1N1R2}{R}{2}{Beati qui habitant}
\gscore{E1N1R3}{R}{3}{Debuit per omnia}
\nocturn{2}
\gscore{E1N2R1}{R}{4}{Ego autem mendicus}
\gscore{E1N2R2}{R}{5}{Vulpes foveas}
\pagebreak
\gscore{E1N2R3}{R}{6}{Cum in forma Dei}
\nocturn{3}
\gscore{E1N3R1}{R}{7}{Vere tu es Rex}
\gscore{E1N3R2}{R}{8}{Sicut per inoboedientiam}
\pagebreak

\feast{E0B}{Infra Octavam Epiphaniæ}
	{Proprium de Tempore}{Infra Octavam Epiphaniæ}{2}{7 usque ad 12 Januarii}
	{Semiduplex}{IV. classis}{}
	{In die, quando legitur initium Epistolæ \Rnum{1} ad Corinthios, \rrrub 1 ut in Festo. 
	 In aliis diebus infra Octavam, \rrrub 1 infra.
	 Alia Responsoria ut in Festo.
	}
	{Die 12 Januarii, \rrrub 1 ut in Festo. In aliis diebus, \rrrub 1 infra.
	 In Feriis, \rrrub 2 ut in Festo. In Festis \Rnum{3}. classis, et in Sabbato S. Mariæ, \respref{2}{0106N1R3}{}
	}
\gscore{0106N1R1b}{R}{1}{Tria sunt munera}

\feast{0113}{In Octava Epiphaniæ,\\ et Baptismate Domini}
	{Proprium de Tempore}{In Octava Epiphaniæ}{2}{13 Januarii}
	{Duplex majus}{II. classis}{Jesu Christi, Domini nostri!Epiphania, octava}
	{}
	{}
\rubric{Responsoria ut in Festo Epiphaniæ.}

\vspace{\baselineskip}

\rubric{In primo die post Octavam, in qua sumendæ sint Lectiones de Epistola \Rnum{1} ad Corinthios, sumuntur Responsoria ut in \Rnum{1} Nocturno Dominicarum post Epiphaniam, ut infra.}

\end{document}