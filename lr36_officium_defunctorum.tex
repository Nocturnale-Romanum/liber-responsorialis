% !TEX TS-program = lualatex
% !TEX encoding = UTF-8

\documentclass[liber-responsorialis.tex]{subfiles}

\ifcsname preamble@file\endcsname
  \setcounter{page}{\getpagerefnumber{M-lr36_officium_defunctorum}}
\fi

\begin{document}
\feast{ODEF}{Officium Defunctorum}
	{Officium Defunctorum}{Officium Defunctorum}{2}{}
	{}{}{Defunctorum!Officium}
	{}{}
\addcontentsline{toc}{section}{Officium Defunctorum}

\nocturn{1}
\smalltitle{Pro Dominica, Feria \Rnum{2} et \Rnum{5}}
\gscore{ODEFN1R1}{R}{1}{Credo quod redemptor}
\gscore{ODEFN1R2}{R}{2}{Qui Lazarum}
\gscore{ODEFN1R3}{R}{3}{Domine quando veneris}
\nocturn{2}
\smalltitle{Pro Dominica, Feria \Rnum{2} et \Rnum{5}}
\gscore{ODEFN2R1}{R}{4}{Memento mei Deus}
\gscore{ODEFN2R2}{R}{5}{Heu mihi Domine}
\gscore{ODEFN2R3}{R}{6}{Ne recorderis peccata}
\nocturn{3}
\smalltitle{Pro Dominica, Feria \Rnum{2} et \Rnum{5}}
\gscore{ODEFN3R1}{R}{7}{Peccantem me quotidie}
\gscore{ODEFN3R2}{R}{8}{Domine secundum actum meum}
\rubric{Sequens Responsorium tunc ponitur, quando tertius tantum Nocturnus dictus fuerit pro Defunctis.}
\gscore{ODEFN3R3a}{R}{9}{Libera me Domine de viis}
\rubric{Sequens Responsorium ponitur loco præcedentis, quando dicti fuerint pro Defunctis tres Nocturni.}
\gscore{ODEFN3R3b}{R}{9}{Libera me Domine de morte}

\end{document}
