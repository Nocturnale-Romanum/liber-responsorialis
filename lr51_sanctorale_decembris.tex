% !TEX TS-program = lualatex
% !TEX encoding = UTF-8

\documentclass[liber-responsorialis_hiemalis.tex]{subfiles}

\ifcsname preamble@file\endcsname
  \setcounter{page}{\getpagerefnumber{M-lr51_sanctorale_decembris}}
\fi

\begin{document}

\feast{1100}{Festa Novembris}{Proprium Sanctorum}{Festa Novembris}{1}{}{}{}{}{}{}

\feast{1129}{S. Saturnini Martyris}
	{Proprium Sanctorum}{Festa Novembris}{2}{29 Novembris (in Adventu)}
	{Commemoratio}{Commemoratio}{Saturnini}
	{In Adventu de Vigilia S.\ Andreæ Apostoli nihi fit Officio, sed fit Officium de Feria cum commemoratio S. Saturnini ad Laudes tantum.}
	{Commemoratio ad Laudes tantum, etiam extra Adventu.}
\rubric{\respdetemp}

\feast{1130}{S. Andreæ Apostoli}
	{Proprium Sanctorum}{Festa Novembris}{2}{30 Novembris}
	{Duplex II. classis}{II. classis}{Andreæ Apostoli}
	{}
	{}
\nocturn{1}
\gscore{1130N1R1}{R}{1}{Cum deambularet}
\gscore{1130N1R2}{R}{2}{Mox ut vocem}
\gscore{1130N1R3}{R}{3}{Doctor bonus et amicus Dei Andreas}
\nocturn{2}
\gscore{1130N2R1}{R}{4}{Homo Dei ducebatur}
\gscore{1130N2R2}{R}{5}{O Bona Crux}
\gscore{1130N2R3}{R}{6}{Expandi manus meas}
\nocturn{3}
\gscore{1130N3R1}{R}{7}{Oravit sanctus Andreas}
\gscore{1130N3R2}{R}{8}{Videns crucem Andreas}

\feast{1200}{Festa Decembris}{Proprium Sanctorum}{Festa Decembris}{1}{}{}{}{}{}{}

\feast{1202}{S. Bibianæ Virginis et Martyris}
	{Proprium Sanctorum}{Festa Decembris}{2}{2 Decembris}
	{Semiduplex}{III. classis}{Bibianæ}
	{\vmrubric}
	{\respdetemp}

\feast{1203}{S. Francisci Xaverii Confessoris}
	{Proprium Sanctorum}{Festa Decembris}{2}{3 Decembris}
	{Duplex majus m.t.v.}{III. classis}{Francisci Xaverii}
	{\conprubric}
	{\respdetemp}

\feast{1204}{S. Petri Chrysologi Episcopi et Ecclesiæ Doctoris}
	{Proprium Sanctorum}{Festa Decembris}{2}{4 Decembris}
	{Duplex}{III. classis}{Petri Chrysologi}
	{\copodorubric}
	{\respdetemp}

\feast{1205}{S. Sabbæ Abbatis}
	{Proprium Sanctorum}{Festa Decembris}{2}{5 Decembris}
	{Commemoratio}{Commemoratio}{Sabbæ}
	{}
	{}
\rubric{\respdetemp}

\feast{1206}{S. Nicolai Episcopi et Confessoris}
	{Proprium Sanctorum}{Festa Decembris}{2}{6 Decembris}
	{Duplex}{III. classis}{Nicolai}
	{\coporubric}
	{\respdetemp}

\feast{1207}{S. Ambrosii Episcopi et Ecclesiæ Doctoris}
	{Proprium Sanctorum}{Festa Decembris}{2}{7 Decembris}
	{Duplex m.t.v.}{III. classis}{Ambrosii}
	{\copodorubric}
	{\respdetemp}

\feast{1208}{In Conceptione Immaculata\\Beatæ Mariæ Virginis}{Proprium Sanctorum}
	{Festa Decembris}{1}{8 Decembris}
	{Duplex I. classis}{I. classis}{Mariæ!Conceptio Immaculata}
	{}
	{}
\nocturn{1}
\gscore{1208N1R1}{R}{1}{Per unum hominem}
\gscore{1208N1R2}{R}{2}{Transite ad me}
\gscore{1208N1R3}{R}{3}{Electa mea candida}
\nocturn{2}
\gscore{1208N2R1}{R}{4}{Ego ex ore Altissimi}
\gscore{1208N2R2}{R}{5}{Nihil inquinatum}
\gscore{1208N2R3}{R}{6}{Signum magnum (sine alleluia) V Induit}
\nocturn{3}
\gscore{1208N3R1}{R}{7}{Hortus conclusus}
\gscore{1208N3R2}{R}{8}{Magnificat V Ecce enim}

\twocolrubric{
	Infra Octavam et in die Octava, Responsoria in \Rnum{1} Nocturno de Tempore, in \Rnum{2} et \Rnum{3} Nocturnis ut in Festo.
	}{
	(Non habet Octavam.)
	}

\feast{1211}{S. Damasii Papæ et Confessoris}
	{Proprium Sanctorum}{Festa Decembris}{2}{11 Decembris}
	{Semiduplex}{III. classis}{Damasii}
	{\coporubric}
	{\respdetemp}


\feast{1213}{S. Luciæ Virginis et Martyris}
	{Proprium Sanctorum}{Festa Decembris}{2}{13 Decembris}
	{Duplex}{III. classis}{Luciæ}
	{Responsoria ut infra}
	{\respdetemp}
\nocturn{1}
\rubric{\respdetemp}
\nocturn{2}
\gscore{1213N2R1}{R}{4}{Lucia Virgo}
\gscore{1213N2R2}{R}{5}{Rogavi Dominum}
\gscore{1213N2R3}{R}{6}{Grata facta est}
\nocturn{3}
\respref{7}{MUXXN2R1}{}
\resprefcumgp{8}{MUXXN2R2}

\feast{1216}{S. Eusebii Episcopi et Martyris}
	{Proprium Sanctorum}{Festa Decembris}{2}{16 Decembris}
	{Semiduplex}{III. classis}{Eusebii}
	{\umexrubric}
	{\respdetemp}

\feast{1221}{S. Thomæ Apostoli}
	{Proprium Sanctorum}{Festa Decembris}{2}{21 Decembris}
	{Duplex II. classis}{II. classis}{Thomæ Apostoli}
	{}
	{}
\rubric{\apexrubric}
	
\end{document}