% !TEX TS-program = lualatex
% !TEX encoding = UTF-8

\documentclass[liber-responsorialis_sanctorale.tex]{subfiles}

\ifcsname preamble@file\endcsname
  \setcounter{page}{\getpagerefnumber{M-lr51_sanctorale_decembris}}
\fi

\begin{document}

\rubric{In omnibus Festis novem Lectionum Domini, beatæ Mariæ Virginis, Angelorum, sancti Joannis Baptistæ, sancti Joseph, Apostolorum, Evangelistarum, necnon in omnibus Duplicibus \Rnum{1} et \Rnum{2} classis, integrum Officium dicitur ut in Proprio vel Communi.}

\rubric{In reliquis vero Festis novem Lectionum, nisi propria suis locis assignentur, Responsoria \Rnum{1} Nocturni de Tempore; si Lectiones de Scriptura occurenti desint, Responsoria \Rnum{1} Nocturni de Communi, nisi aliter notatur; reliqua de Communi.}

\rubric{Per Octavas autem communes, nisi propria suis locis assignentur, Responsoria \Rnum{1} Nocturni de Tempore; si Lectiones de Scriptura occurenti desint, Responsoria \Rnum{1} Nocturni ut in Festo, nisi aliter notatur; reliqua ut in Festo.}

\rubric{In Festis autem et diebus Octavis simplicibus, Responsoria de Tempore, addito \Rnum{2} Responsorio \normaltext{Glória Patri}.}

\feast{1200}{Festa Decembris}{Proprium Sanctorum}{Festa Decembris}{1}{}{}{}{}{}{}
\addcontentsline{toc}{section}{Festa Decembris}

\feast{1202}{S. Bibianæ Virginis et Martyris}
	{Proprium Sanctorum}{Festa Decembris}{2}{2 Decembris}
	{Semiduplex}{III. classis}{Bibianæ}
	{\vmrubric}
	{\respdetemp}

\feast{1203}{S. Francisci Xaverii Confessoris}
	{Proprium Sanctorum}{Festa Decembris}{2}{3 Decembris}
	{Duplex majus m.t.v.}{III. classis}{Francisci Xaverii}
	{\conprubric}
	{\respdetemp}

\feast{1204}{S. Petri Chrysologi Episcopi et Ecclesiæ Doctoris}
	{Proprium Sanctorum}{Festa Decembris}{2}{4 Decembris}
	{Duplex}{III. classis}{Petri Chrysologi}
	{\copodorubric}
	{\respdetemp}

\feast{1205}{S. Sabbæ Abbatis}
	{Proprium Sanctorum}{Festa Decembris}{2}{5 Decembris}
	{Commemoratio}{Commemoratio}{Sabbæ}
	{}
	{}
\rubric{\respdetemp}

\feast{1206}{S. Nicolai Episcopi et Confessoris}
	{Proprium Sanctorum}{Festa Decembris}{2}{6 Decembris}
	{Duplex}{III. classis}{Nicolai}
	{\coporubric}
	{\respdetemp}

\feast{1207}{S. Ambrosii Episcopi et Ecclesiæ Doctoris}
	{Proprium Sanctorum}{Festa Decembris}{2}{7 Decembris}
	{Duplex m.t.v.}{III. classis}{Ambrosii}
	{\copodorubric}
	{\respdetemp}

\feast{1208}{In Conceptione Immaculata\\Beatæ Mariæ Virginis}{Proprium Sanctorum}
	{Festa Decembris}{1}{8 Decembris}
	{Duplex I. classis}{I. classis}{Mariæ!Conceptio Immaculata}
	{}
	{}
\nocturn{1}
\gscore{1208N1R1}{R}{1}{Per unum hominem}
\gscore{1208N1R2}{R}{2}{Transite ad me}
\gscore{1208N1R3}{R}{3}{Electa mea candida}
\nocturn{2}
\gscore{1208N2R1}{R}{4}{Ego ex ore Altissimi}
\gscore{1208N2R2}{R}{5}{Nihil inquinatum}
\gscore{1208N2R3}{R}{6}{Signum magnum\idxnewline\vvrub Induit}
\nocturn{3}
\gscore{1208N3R1}{R}{7}{Hortus conclusus}
\gscore{1208N3R2}{R}{8}{Magnificat}

\twocolrubric{
	Infra Octavam et in die Octava, Responsoria in \Rnum{1} Nocturno de Tempore, in \Rnum{2} et \Rnum{3} Nocturnis ut in Festo.
	}{
	(Non habet Octavam.)
	}

\feast{1211}{S. Damasii Papæ et Confessoris}
	{Proprium Sanctorum}{Festa Decembris}{2}{11 Decembris}
	{Semiduplex}{III. classis}{Damasii}
	{\coporubric}
	{\respdetemp}


\feast{1213}{S. Luciæ Virginis et Martyris}
	{Proprium Sanctorum}{Festa Decembris}{2}{13 Decembris}
	{Duplex}{III. classis}{Luciæ}
	{Responsoria ut infra}
	{\respdetemp}
\nocturn{1}
\rubric{\respdetemp}
\nocturn{2}
\gscore{1213N2R1}{R}{4}{Lucia Virgo}
\gscore{1213N2R2}{R}{5}{Rogavi Dominum}
\gscore{1213N2R3}{R}{6}{Grata facta est}
\nocturn{3}
\respref{7}{MUXXN2R1}{}
\resprefcumgp{8}{MUXXN2R2}

\feast{1216}{S. Eusebii Episcopi et Martyris}
	{Proprium Sanctorum}{Festa Decembris}{2}{16 Decembris}
	{Semiduplex}{III. classis}{Eusebii}
	{\umexrubric}
	{\respdetemp}

\feast{1221}{S. Thomæ Apostoli}
	{Proprium Sanctorum}{Festa Decembris}{2}{21 Decembris}
	{Duplex II. classis}{II. classis}{Thomæ Apostoli}
	{}
	{}
\rubric{\apexrubric}

%%%%%%%%%% Comment out this section when those feasts are in the Temporale

\feast{1226}{In Festo S. Stephani Protomartyris}
	{Proprium de Tempore}{S. Stephani Protomartyris}{2}{26 Decembris}
	{Duplex II. classis}{II. classis}{Stephani}
	{}
	{}
\nocturn{1}
\gscore{1226N1R1}{R}{1}{Stephanus autem}
\gscore{1226N1R2}{R}{2}{Videbant omnes Stephanum}
\pagebreak
\gscore{1226N1R3}{R}{3}{Intuens in caelum}
\nocturn{2}
\gscore{1226N2R1}{R}{4}{Lapidabant Stephanum}
\pagebreak
\gscore{1226N2R2}{R}{5}{Impetum fecerunt}
\pagebreak
\gscore{1226N2R3}{R}{6}{Impii super justum}
\vspace{\baselineskip}
\nocturn{3}
{
	\grechangedim{baselineskip}{75pt plus 3pt minus 0pt}{scalable}
	\gscore{1226N3R1}{R}{7}{Stephanus servus}
	\sixlinesvspace
	\fiveplustitlevspace
	\fiveplustitlevspace
	\gscore{1226N3R2}{R}{8}{Patefactae sunt}
	\sixlinesvspace
}
\pagebreak

\feast{1227}{In Festo S. Joannis\\Apostoli et Evangelistæ}
	{Proprium de Tempore}{S. Joannis Apostoli et Evangelistæ}{2}{27 Decembris}
	{Duplex II. classis}{II. classis}{Joannis Evangelistæ}
	{}
	{}
\nocturn{1}
\gscore{1227N1R1}{R}{1}{Valde honorandus}
\gscore{1227N1R2}{R}{2}{Hic est discipulus}
\gscore{1227N1R3}{R}{3}{Hic est beatissimus}
\nocturn{2}
\gscore{1227N2R1}{R}{4}{Qui vicerit faciam}
\gscore{1227N2R2}{R}{5}{Diligebat autem eum}
\pagebreak
\gscore{1227N2R3}{R}{6}{In medio ecclesiae}
\pagebreak
\nocturn{3}
\gscore{1227N3R1}{R}{7}{In illo die suscipiam}
\fiveplustitlevspace
\pagebreak
\gscore{1227N3R2}{R}{8}{Iste est Joannes}
\pagebreak

\fiveplustitlevspace
\feast{1228}{In Festo Ss. Innocentium Martyrum}
	{Proprium de Tempore}{Ss. Innocentium Martyrum}{2}{28 Decembris}
	{Duplex II. classis}{II. classis}{Innocentium}
	{}
	{}
\fiveplustitlevspace
\nocturn{1}
\gscore{1228N1R1}{R}{1}{Centum quadraginta}
\gscore{1228N1R2}{R}{2}{Sub altare Dei audivi}
\sixlinesvspace
\pagebreak
\gscore{1228N1R3}{R}{3}{Adoraverunt}
\pagebreak
\nocturn{2}
\gscore{1228N2R1}{R}{4}{Effuderunt sanguinem}
\pagebreak
\gscore{1228N2R2}{R}{5}{Isti sunt sancti qui passi}
\sixlinesvspace
\sixlinesvspace
\pagebreak
\gscore{1228N2R3}{R}{6}{Isti sunt sancti qui non}
\nocturn{3}
\gscore{1228N3R1}{R}{7}{Cantabant sancti}
\pagebreak
\rubric{Nisi Festum Sanctorum Innocentium venerit in Dominica, \vvrub \normaltext{Glória Patri} omittitur.}
\gscore{1228N3R2}{R}{8}{Vidi sub altare}
\sixlinesvspace
\sixlinesvspace
\sixlinesvspace
\rubric{Si Festum Sanctorum Innocentium venerit in Dominica, loco \rrrub 9. \normaltext{Te Deum} cantatur.}
\gscore{1228N3R3}{R}{9}{Isti qui amicti sunt}
\pagebreak

\feast{1229}{S. Thomæ Cantuariensis Episcopi et Martyris}
	{Proprium de Tempore}{S. Thomæ Episcopi et Martyris}{2}{29 Decembris}
	{Duplex}{Commemoratio}{Thomæ Cantuariensis}
	{Responsoria \Rnum{1} Nocturni ut in Dominica infra Octavam Nativitatis.
		Responsoria \Rnum{2} et \Rnum{3} Nocturnorum ut in Communi unius Martyris pag.\ \pageref{M-UMEXN2R1}. 
	}
	{Responsoria 1. et 2. de Dominica infra Octavam Nativitatis.}

\feast{1231}{S. Silvestri Papæ et Confessoris}
	{Proprium de Tempore}{S. Silvestri Papæ et Confessoris}{2}{31 Decembris}
	{Duplex}{Commemoratio}{Silvestri Papæ}
	{Responsoria \Rnum{1} Nocturni ut in Dominica infra Octavam Nativitatis.
		Responsoria \Rnum{2} et \Rnum{3} Nocturnorum ut in Communi Confessoris Pontificis pag.\ \pageref{M-COPON2R1}. 
	}
	{Responsoria 1. et 2. de Dominica infra Octavam Nativitatis.}

\end{document} 