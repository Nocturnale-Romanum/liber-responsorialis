% !TEX TS-program = lualatex
% !TEX encoding = UTF-8

\documentclass[liber-responsorialis.tex]{subfiles}

\ifcsname preamble@file\endcsname
  \setcounter{page}{\getpagerefnumber{M-lr47_sanctorale_augusti}}
\fi

\begin{document}

\feast{0800}{Festa Augusti}{Proprium Sanctorum}{Festa Augusti}{1}{}{}{}{}{}{}

\feast{0801}{S. Petri ad Vincula}
	{Proprium Sanctorum}{Festa Augusti}{2}{1 Augusti}
	{Duplex majus}{(Omittitur)}{Petri Apostoli!ad Vincula}
	{}
	{}
\rubric{Omnia de Communi Apostolorum pag.\ \pageref{M-APEX}, præter Hymnus et Responsoria.
Hymnus \scorename{0118H}, pag.\ \pageref{M-0118H}.}
\nocturn{1}
\rubric{Trium Responsoria ut in \Rnum{1} Nocturno Cathedræ S.\ Petri Apost.\ Romæ pag.\ \pageref{M-0118N1R1}.}
\nocturn{2}
\respref{4}{0629N2R1}{}
\gscore{0801N2R2}{R}{5}{Surge Petre}
\resprefcumgp{6}{0118N2R1}
\nocturn{3}
\respref{7}{0118N2R2}{}
\resprefcumgp{8}{0118N3R1}
\tedeumrubric


\feast{0802}{S. Alfonsi Mariæ de Ligorio\\Episcopi Confessoris et Ecclesiæ Doctoris}
	{Proprium Sanctorum}{Festa Augusti}{2}{2 Augusti}
	{Duplex m.t.v.}{III. classis}{Alfonsi Mariæ de Ligorio}
	{Commune Doctorum Ecclesiæ, pag.\ \pageref{M-CODO}.}
	{}

\feast{0803}{In Inventione S. Stephani Protomartyris}
	{Proprium Sanctorum}{Festa Augusti}{2}{3 Augusti}
	{Semiduplex}{(Omittitur)}{Stephani!Inventio}
	{Invitatorium et Hymnus de Communi unius Martyris pag.\ \pageref{M-UMEX}. Antiphonæ, Psalmi et Versus Nocturnorum de Feria. Responsoria ut in Festo, pag.\ \pageref{M-1226}.}
	{}

\feast{0804}{S. Dominici Confessoris}
	{Proprium Sanctorum}{Festa Augusti}{2}{4 Augusti}
	{Duplex majus m.t.v.}{III. classis}{Dominici}
	{Commune Confessoris non Pontificis, pag.\ \pageref{M-CONP}.}
	{}

\feast{0805}{In Dedicatione Sanctæ Mariæ ad Nives}
	{Proprium Sanctorum}{Festa Augusti}{2}{5 Augusti}
	{Duplex majus}{III. classis}{Mariæ!Dedicatio S. M. ad Nives}
	{Omnia de Communi B.M.V., pag.\ \pageref{M-CBMV}.}
	{}

\feast{0806}{In Transfiguratione\\Domini nostri Jesu Christi}
	{Proprium Sanctorum}{Festa Augusti}{2}{6 Augusti}
	{Duplex II. classis}{II. classis}{Jesu Christi, Domini nostri!Transfiguratio}
	{}
	{}
\gscore{0806I}{I}{}{Summum Regem gloriae Christum adoremus}
\gscore{0806H}{H}{}{Quicumque Christum quaeritis}
\nocturn{1}
\gscore{0806N1A1}{A}{1}{Paulo minus ab Angelis minoratus}
\psalmus{8}{}
\gscore{0806N1A2}{A}{2}{Revelavit Dominus condensa}
\psalmus{28}{}
\gscore{0101N2A1}{A}{3}{Speciosus forma}
\psalmus{44}{}
\versiculus{Gloriósus apparuísti in conspéctu Dómini.}{Proptérea decórem induit te Dóminus.}
\gscore{0806N1R1}{R}{1}{Surge illuminare Jerusalem}
\gscore{0806N1R2}{R}{2}{In splendenti nube}
\gscore{0806N1R3}{R}{3}{Videte qualem charitatem}
\nocturn{2}
\gscore{0806N2A1}{A}{4}{Illuminans tu mirabiliter}
\psalmus{75}{}
\gscore{0806N2A2}{A}{5}{Melior est dies una in atriis tuis}
\psalmus{83}{}
\gscore{0806N2A3}{A}{6}{Gloriosa dicta sunt de te civitas Dei}
\psalmus{86}{}
\versiculus{Glória et honóre coronásti eum, Dómine.}{Et Constituísti eum super ópera mánuum tuárum.}
\gscore{0806N2R1}{R}{4}{Inebriati sunt ab ubertate}
\gscore{0806N2R2}{R}{5}{Praeceptor bonum est}
\gscore{0806N2R3}{R}{6}{Si ministratio mortis}
\nocturn{3}
\gscore{0806N3A1}{A}{7}{Thabor et Hermon in nomine tuo}
\psalmus{88}{}
\gscore{0806N3A2}{A}{8}{Lux orta est justo et rectis corde laetitia}
\psalmus{96}{}
\gscore{0806N3A3}{A}{9}{Confessionem et decorem induit}
\psalmus{103}{}
\versiculus{Magna est glória ejus in salutári tuo.}{Glóriam et magnum decórem impónes super eum.}
\gscore{0806N3R1}{R}{7}{Vocavit nos deus vocatione sua}
\gscore{0806N3R2}{R}{8}{Deus qui fecit de tenebris lumen}
\tedeumrubric

\feast{0807}{S. Cajetani Confessoris}
	{Proprium Sanctorum}{Festa Augusti}{2}{7 Augusti}
	{Duplex}{III. classis}{Cajetani}
	{Commune Confessoris non Pontificis, pag.\ \pageref{M-CONP}.}
	{}

\feast{0808}{Ss. Cyriaci, Largi et Smaragdi Martyrum}
	{Proprium Sanctorum}{Festa Augusti}{2}{8 Augusti}
	{Semiduplex}{(Omittitur)}{Cyriaci, Largi et Smaragdi}
	{Commune plurimorum Martyrum, pag.\ \pageref{M-PMEX}.}
	{}

\feast{0809}{S. Joannis Mariæ Vianney Confessoris}
	{Proprium Sanctorum}{Festa Augusti}{2}{9 Augusti}
	{Duplex m.t.v.}{(Omittitur : cf. 8 Augusti)}{Joannis Mariæ Vianney}
	{Commune Confessoris non Pontificis, pag.\ \pageref{M-CONP}.}
	{}

\feast{0809b}{In Vigilia S. Laurentii}
	{Proprium Sanctorum}{Festa Augusti}{2}{9 Augusti}
	{(Omittitur)}{III. classis}{Laurentii!Vigilia}
	{Officium fit de Feria, ut in Ordinario et Psalterio, præter Lectiones. Responsoria de Feria.}
	{}

\feast{0810}{S. Laurentii Martyris}
	{Proprium Sanctorum}{Festa Augusti}{2}{10 Augusti}
	{Duplex II. classis}{II. classis}{Laurentii}
	{Cum octava simplici. Hymnus et Psalmi de Communi unius Martyris pag.\ \pageref{M-UMEX}, præter Psalmus octavus, ut infra. Reliqua ut infra.}
	{}
\gscore{0810I}{I}{}{Beatus Laurentius Christi Martyr triumphat}
\nocturn{1}
\gscore{0810N1A1}{A}{1}{Quo progrederis sine filio}
\gscore{0810N1A2}{A}{2}{Non me derelinque pater sancte}
\gscore{0810N1A3}{A}{3}{Non ego te desero fili}
\versiculus{Glória et honóre coronásti eum, Dómine.}{Et constituísti eum super ópera mánuum tuárum.}
\gscore{0810N1R1}{R}{1}{Levita Laurentius}
\gscore{0810N1R2}{R}{2}{Puer meus noli timere qui}
\gscore{0810N1R3}{R}{3}{Strinxerunt corporis}
\nocturn{2}
\gscore{0810N2A1}{A}{4}{Beatus Laurentius orabat}
\gscore{0810N2A2}{A}{5}{Dixit Romanus ad beatum Laurentium}
\gscore{0810N2A3}{A}{6}{Beatus Laurentius dixit}
\versiculus{Posuísti, Dómine, super caput ejus.}{Corónam de lápide pretióso.}
\gscore{0810N2R1}{R}{4}{Quo progrederis sine filio}
\gscore{0810N2R2}{R}{5}{Non me derelinque}
\gscore{0810N2R3}{R}{6}{Beatus Laurentius clamavit}
\nocturn{3}
\gscore{0810N3A1}{A}{7}{Strinxerunt corporis}
\gscore{0810N3A2}{A}{8}{Igne me examinasti}
\psalmus{16}{}
\gscore{0810N3A3}{A}{9}{Interrogatus te Dominum confessus sum}
\versiculus{Magna est glória ejus in salutári tuo.}{Glóriam et magnum decórem impónes super eum.}
\gscore{0810N3R1}{R}{7}{In craticula te Deum non negavi}
\gscore{0810N3R2}{R}{8}{O Hippolyte si credis}
\tedeumrubric

\feast{0811}{Ss. Tiburtii et Susannæ Martyrum}
	{Proprium Sanctorum}{Festa Augusti}{2}{11 Augusti}
	{Simplex}{Commemoratio}{Tiburtii et Susannæ}
	{Commune plurimorum Martyrum, pag.\ \pageref{M-PMEX}.}
	{}

\feast{0812}{S. Claræ Virginis}
	{Proprium Sanctorum}{Festa Augusti}{2}{12 Augusti}
	{Duplex}{III. classis}{Claræ}
	{Commune Virginum, pag.\ \pageref{M-MU}.}
	{}

\feast{0813}{Ss. Hippolyti et Cassiani Martyrum}
	{Proprium Sanctorum}{Festa Augusti}{2}{13 Augusti}
	{Simplex}{Commemoratio}{Hippolyti et Cassiani}
	{Commune plurimorum Martyrum, pag.\ \pageref{M-PMEX}.}
	{}

\feast{0814}{In Vigilia Assumptionis\\Beatæ Mariæ Virginis}
	{Proprium Sanctorum}{Festa Augusti}{2}{14 Augusti}
	{Simplex}{II. classis}{Mariæ!Assumptio, vigilia}
	{Officium fit de Feria, ut in Ordinario et Psalterio, præter Lectiones. Responsoria de Feria.}
	{}

\feast{0815}{In Assumptione Beatæ Mariæ Virginis}
	{Proprium Sanctorum}{Festa Augusti}{1}{15 Augusti}
	{Duplex I. classis}{I. classis}{Mariæ!Assumptio}
	{Cum Octava communi. Omnia de Communi Festorum Beatæ Mariæ Virginis, pag.\ \pageref{M-CBMV}, præter quæ habentur propria.}
	{}
\gscore{0815I}{I}{}{Venite adoremus Regem regum... assumpta}
\rubric{Secundum antiquiorem consuetudinem, hymnus \scorename{CBMVH}, pag.\ \pageref{M-CBMVH}, vel sequens.}
\gscore{0815H}{H}{}{Surge jam terris}
\nocturn{1}
\rubric{Tres Psalmi de Communi.}
\gscore{0815N1A1}{A}{1}{Exaltata est sancta}
\gscore{0815N1A2}{A}{2}{Paradisi portae}
\gscore{0815N1A3}{A}{3}{Benedicta tu in mulieribus}
\versiculus{Exaltáta est sancta Dei Génitrix.}{Super choros Angelórum ad cœléstia regna.}
\gscore{0815N1R1}{R}{1}{Vidi speciosam sicut V Quae est ista}
\respref{2}{CBMVN2R1}{}
\resprefcumgp{3}{CBMVN2R2}
\nocturn{2}
\rubric{Antiphonæ et Psalmi de Communi.}
\versiculus{Assúmpta est Maria in cœlum: gaudent Angeli.}{Laudántes benedícunt Dóminum.}
\resprefsinegp{4}{CBMVN2R3}
\resprefsinegp{5}{CBMVN3R2}
\gscore{0815N2R3}{R}{6}{Beata es Maria quae}
\nocturn{3}
\rubric{Antiphonæ et Psalmi de Communi.}
\versiculus{María Virgo assúmpta est ad æthéreum thálamum.}{In quo Rex regum stelláto sedet sólio.}
\gscore{0815N3R1}{R}{7}{Diffusa est V Myrrha}
\gscore{0815N3R2}{R}{8}{Beata es Virgo Maria Dei Genitrix}
\tedeumrubric \rubric{Infra Octavam et in die Octava Antiphonæ et Psalmi ad omnes Horas et Versus Nocturnorum de occurenti hebdomadæ die, ut in Psalterio, reliqua ut in Festo præter Lectiones, quæ in \Rnum{1} Nocturno dicuntur de Scriptura occurenti cum suis Responsoriis de Tempore.}

\feast{0816}{S. Joachim Confessoris\\Patris Beatæ Mariæ Virginis}
	{Proprium Sanctorum}{Festa Augusti}{2}{16 Augusti}
	{Duplex II. classis}{II. classis}{Joachim}
	{Commune Confessoris non Pontificis, pag.\ \pageref{M-CONP}.}
	{}

\feast{0817}{S. Hyacinthi Confessoris}
	{Proprium Sanctorum}{Festa Augusti}{2}{17 Augusti}
	{Duplex m.t.v.}{III. classis}{Hyacinthi}
	{Commune Confessoris non Pontificis, pag.\ \pageref{M-CONP}.}
	{}

\feast{0818}{Die IV infra Octavam Assumptionis}
	{Proprium Sanctorum}{Festa Augusti}{2}{18 Augusti}
	{Semiduplex}{(Omittitur)}{}
	{Antiphonæ, Psalmi et Versus Nocturnorum de Feria, Invitatorium et Hymnus ut in Festo, Responsoria in \Rnum{1} Nocturno de Feria occurenti, in \Rnum{2} et \Rnum{3} Nocturno ut in Festo.}
	{}

\feast{0819}{S. Joannis Eudes Confessoris}
	{Proprium Sanctorum}{Festa Augusti}{2}{19 Augusti}
	{Duplex}{III. classis}{Joannis Eudes}
	{Commune Confessoris non Pontificis, pag.\ \pageref{M-CONP}.}
	{}

\feast{0820}{S. Bernardi Abbatis et Ecclesiæ Doctoris}
	{Proprium Sanctorum}{Festa Augusti}{2}{20 Augusti}
	{Duplex}{III. classis}{Bernardi}
	{Commune Doctorum Ecclesiæ, pag.\ \pageref{M-CODO}.}
	{}

\feast{0821}{S. Joannæ Franciscæ Frémiot de Chantal Viduæ}
	{Proprium Sanctorum}{Festa Augusti}{2}{21 Augusti}
	{Duplex}{III. classis}{Joannæ de Chantal}
	{Commune non Virginum, pag.\ \pageref{M-MU}.}
	{}

\feast{0822}{In Festo Immaculati Cordis Beatæ Mariæ Virginis}
	{Proprium Sanctorum}{Festa Augusti}{2}{22 Augusti}
	{Duplex II. classis}{II. classis}{Mariæ!Cor Immaculatum}
	{Omnia ut in Communi Festorum B.M.V.\ pag.\ \pageref{M-CBMV}.}
	{}

\feast{0823}{S. Philippi Benitii Confessoris}
	{Proprium Sanctorum}{Festa Augusti}{2}{23 Augusti}
	{Duplex m.t.v.}{III. classis}{Philippi Benitii}
	{Commune Confessoris non Pontificis, pag.\ \pageref{M-CONP}.}
	{}

\feast{0824}{S. Bartholomæi Apostoli}
	{Proprium Sanctorum}{Festa Augusti}{2}{24 Augusti}
	{Duplex II. classis}{II. classis}{Bartholomæi}
	{Omnia de Communi Apostolorum pag.\ \pageref{M-APEX}.}
	{}

\feast{0825}{S. Ludovici Regis Confessoris}
	{Proprium Sanctorum}{Festa Augusti}{2}{25 Augusti}
	{Semiduplex}{III. classis}{Ludovici}
	{Commune Confessoris non Pontificis, pag.\ \pageref{M-CONP}.}
	{}

\feast{0826}{S. Zephyrini Papæ et Martyris}
	{Proprium Sanctorum}{Festa Augusti}{2}{26 Augusti}
	{Simplex}{Commemoratio}{Zephyrini}
	{Commune unius Martyris, pag.\ \pageref{M-UMEX}.}
	{}

\feast{0827}{S. Josephi Calasanctii Confessoris}
	{Proprium Sanctorum}{Festa Augusti}{2}{27 Augusti}
	{Duplex m.t.v.}{III. classis}{Josephi Calasanctii}
	{Commune Confessoris non Pontificis, pag.\ \pageref{M-CONP}.}
	{}

\feast{0828}{S. Augustini Hipponensis Episcopi\\Confessoris et Ecclesiæ Doctoris}
	{Proprium Sanctorum}{Festa Augusti}{2}{28 Augusti}
	{Duplex}{III. classis}{Augustini Hipponensis}
	{Commune Doctorum Ecclesiæ, pag.\ \pageref{M-CODO}.}
	{}

\feast{0829}{In Decollatione Sancti Joannis Baptistæ}
	{Proprium Sanctorum}{Festa Augusti}{2}{29 Augusti}
	{Duplex majus}{III. classis}{Joannis Baptistæ!Decollatio}
	{Omnia de Communi unius Martyris, pag.\ \pageref{M-UMEX}, præter Responsoria in \Rnum{1} Nocturno, ut infra.}
	{}
\gscore{0829N1R1}{R}{1}{Misit Herodes rex}
\gscore{0829N1R2}{R}{2}{Joannes Baptista arguebat}
\gscore{0829N1R3}{R}{3}{Puellae saltanti}
\tedeumrubric

\feast{0830}{S. Rosæ a Sancta Maria Limanæ Virginis}
	{Proprium Sanctorum}{Festa Augusti}{2}{30 Augusti}
	{Duplex}{III. classis}{Rosæ Limanæ}
	{Commune Virginum, pag.\ \pageref{M-MU}.}
	{}

\feast{0831}{S. Raymundi Nonnati Confessoris}
	{Proprium Sanctorum}{Festa Augusti}{2}{31 Augusti}
	{Duplex m.t.v.}{III. classis}{Raymundi Nonnati}
	{Commune Confessoris non Pontificis, pag.\ \pageref{M-CONP}.}
	{}

\end{document}