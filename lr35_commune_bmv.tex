% !TEX TS-program = lualatex
% !TEX encoding = UTF-8

\documentclass[liber-responsorialis.tex]{subfiles}

\ifcsname preamble@file\endcsname
  \setcounter{page}{\getpagerefnumber{M-lr35_commune_bmv}}
\fi

\begin{document}
\feast{CBMV}{Commune Beatæ Mariæ Virginis}
	{Commune Sanctorum}{Commune Beatæ Mariæ Virginis}{2}{}{}{}{}{}{}
\addcontentsline{toc}{section}{Commune Beatæ Mariæ Virginis}

\nocturn{1}
\gscore{CBMVN1R1}{R}{1}{Sancta et Immaculata V Benedicta tu}
\gscore{CBMVN1R2}{R}{2}{Congratulamini...parvula V Beatam me dicent}
\gscore{CBMVN1R3}{R}{3}{Beata es Maria V Ave Maria V Gloria}
\nocturn{2}
\gscore{CBMVN2R1}{R}{4}{Sicut cedrus}
\gscore{CBMVN2R2}{R}{5}{Quae est ista quae processit}
\gscore{CBMVN2R3}{R}{6}{Ornatam in monilibus}
\nocturn{3}
\gscore{CBMVN3R1}{R}{7}{Felix namque es}
\vspace{-0.25in}
\begin{tabbing}
{\hskip 0.5em}\={\hskip 1.8em}\={\hskip 2.2em}\={\hskip 2.0em}\={\hskip 1.5em}\={\hskip 1.8em}\={\hskip 2em}\={\hskip 1.7em}\={\hskip 3.2em}\={\hskip 2.8em}\=\\
%\>1\>2\>3\>4\>5\>6\>7\>8\>9\>0\\
\>\>\>\>{\small sanctam}\>\>Fe-\>sti-\>vi-\>tá-\>tem. \\
\>\>\>\>\>Vi-\>si-\>ta-\>ti-\>ó-\>nem.\\
\>{\small so{\bfseries lé}mnem}\\
\>\>\>\>\hspace{-0.75em}Com-\>me-\>mo-\>ra-\>ti-\>ó-\>nem.\\
\>\>\>\>\>\>As-\>sum\>pti-\>ó-\>nem.\\
\>\>\>\>\>\>Na-\>ti-\>vi-\>tá-\>tem.\\
\>\underline{\small tui}\>{\small sancti}\>{\small {\bfseries Nó}minis}\\
\>\>\>\>\hspace{-0.75em}Com-\>me-\>mo-\>ra-\>ti-\>ó-\>nem.\\
\>\underline{{\small sanc{\bfseries tís}simi}}\>\>{\small Ro{\bfseries sá}rii}\>\>\>So-\>le-\>mni-\>tá-\>tem.\\
\>{\bfseries san}ctam\>\>\>\>\hspace{-0.1em}Præ-\>sen-\>ta-\>ti-\>ó-\>nem. \\
\>\underline{\small tuum}\>{\small{\ \bfseries san}ctum}\>\>implo-\>\>rant\>au-\>xí-\>li-\>um.\\
\end{tabbing}
\gscore{CBMVN3R2}{R}{8}{Beatam me dicent V Et misericordia ejus V Gloria}

\feast{CSMS}{De Sancta Maria in Sabbato}
	{Commune Sanctorum}{De Sancta Maria in Sabbato}{2}{}
	{}{}{Sabbato Sanctae Mariae}
	{}{}
\addcontentsline{toc}{section}{De Sancta Maria in Sabbato}
\rubric{Responsoria de Tempore. 
	Lectio vero \Rnum{3} propria, juxta ordinem Mensium. 
	Post ultimam Lectionem dicitur Hymnus \emph{Te Deum}.}

\feast{OPBM}{Officium Parvum\\Beatæ Mariæ Virginis}
	{Officium Parvum B. M. V.}{Officium Parvum B. M. V.}{2}{}{}{}{}{}{}
\addcontentsline{toc}{section}{Officium Parvum B. M. V.}

\intermediatetitle{Extra Tempus Adventus}

\respref{1}{CBMVN1R1}{}
\resprefsinegp{2}{CBMVN1R3}
\respref{3}{CBMVN3R1}{cum \vvrub \normaltext{...commemoratiónem} et \vvrub \normaltext{Glória Patri} in fine}
\rubric{Extra Septuagesimam et Quadragesimam omittitur \rrrub 3. et dicitur \normaltext{Te Deum}.}

\intermediatetitle{Tempore Adventus}

\gscore{OPBMR1}{R}{1}{Missus est Gabriel V Dabit ei Dominux}
\gscore{OPBMR2}{R}{2}{Ave Maria gratia plena V Quomodo fiet istud}
\gscore{OPBMR3}{R}{3}{Suscipe verbum V Ut paries}

\end{document}