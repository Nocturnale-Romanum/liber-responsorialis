%%%%%%%%%%%%%%% INDICES %%%%%%%%%%%%%%%

\usepackage{imakeidx}

\indexsetup{level=\section*,toclevel=section,noclearpage,othercode=\footnotesize\thispagestyle{empty}}
\makeindex[name=I,title=Index Invitatoriorum, columns=2,columnseprule]
\makeindex[name=H,title=Index Hymnorum, columns=2,columnseprule]
\makeindex[name=A,title=Index Antiphonarum, columns=2,columnseprule]
\makeindex[name=R,title=Index Responsoriorum, columns=2,columnseprule]
\makeindex[name=P,title=Index Psalmorum, columns=2,columnseprule]
\makeindex[name=T,title=Toni Communes, columns=2,columnseprule]
\makeindex[name=F,title=Index Festorum, columns=2,columnseprule]

%%%%%%%%%%%%%%% STANDARD PACKAGES %%%%%%%%%%%%%%%

%% This is the format of the recent Solesmes books.
%%\usepackage[paperwidth=135mm, paperheight=205mm]{geometry}

%% This reuses the text area width of the recent Solesmes books, with more inner margin, to get an aspect ratio of 1,33
%% which corresponds to the largest KDP hardcover format (8,25 in x 11 in)
\usepackage[paperwidth=150mm, paperheight=205mm]{geometry}

%% This is the format of the 1912 Antiphonale Romanum
%\usepackage[paperwidth=160mm, paperheight=240mm]{geometry}

\usepackage{fontspec}
\usepackage[nolocalmarks]{polyglossia}
\usepackage[table]{xcolor}
\usepackage{fancyhdr}
\usepackage{titlesec}
\usepackage{setspace}
\usepackage{expl3}
\usepackage{hyperref}
\usepackage{refcount}
\usepackage{needspace}
\usepackage{etoolbox}
\usepackage{enumitem}
\usepackage{lettrine}
\usepackage{tocloft}

%%%%%%%%%%%%%%% HYPHENATION AND TYPOGRAPHICAL CONVENTIONS %%%%%%%%%%%%%%

%\setdefaultlanguage[variant=ecclesiastic, hyphenation=liturgical, usej=true, babelshorthands=false]{latin}
\setdefaultlanguage[variant=french, frenchitemlabels=false]{french}
\setotherlanguage{english}
%% this latin option basically boils down to French with slightly thinner pre-punctuation spaces 
%% and slightly different hyphenation (but how different?)
%% for some reason, it produces more underfull hboxes in two-col psalms, than French; and about 0.5% more pages.
%% French is therefore to be kept until further notice.
%\setotherlanguage[variant=ecclesiastic, hyphenation=liturgical, usej=true]{latin}

%%%%%%%%%%%%%%% GEOMETRY %%%%%%%%%%%%%%%

%% This should mimic the layout of the recent Solesmes books.
\geometry{
inner=25mm,
outer=15mm,
top=12mm,
bottom=15mm,
headsep=3mm,
}

%% General scale of all graphical elements.
%% Values different from 1 are largely untested.
%% Used in those commands (e.g. everything FontSpec) that use a scale parameter.
\newcommand{\customscale}{1}

%% Provide the command \fpevalc as a copy of the code-level \fp_eval:n.
%% \fpevalc allows to evaluate floating point calculation for scaled parameters, e.g. \setSomeStretchFactor{\fpevalc{0,6 * \customscale}}
\ExplSyntaxOn
\cs_new_eq:NN \fpevalc \fp_eval:n
\ExplSyntaxOff

%% No indentation of paragraphs
\setlength{\parindent}{0mm}

%% We want to allow large inter-words space 
%% to avoid overfull boxes in two-columns rubrics.
\sloppy

%%%%%%%%%%%%%%% GREGORIO CONFIG %%%%%%%%%%%%%%%

\usepackage[forcecompile]{gregoriotex}

%% this limits how much scores can stretch vertically
%% when they are forced to adhere to the bottom of a page (i.e followed by \pagebreak)
\grechangedim{baselineskip}{65pt plus 1pt minus 5pt}{scalable}

%% text above lines shall be of color gregoriocolor
\grechangestyle{abovelinestext}{\color{gregoriocolor}\footnotesize\itshape}
%% fine-tuning of space beween the staff and the text above lines
\newcommand{\altraise}{-0.4mm} %% default is -0.1cm
\grechangedim{abovelinestextraise}{\altraise}{scalable}
\grechangedim{abovelinestextheight}{10mm}{scalable}

%% fine-tuning of space between the staff and the lyrics
\newcommand{\textraise}{2.8ex} %% default is 3.48471 ex
\grechangedim{spacelinestext}{\textraise}{scalable}

%% fine-tuning of space between the initial and the annotations
\newcommand{\annraise}{0mm} %% default is -0.2mm
\grechangedim{annotationraise}{\annraise}{scalable}

%% fine-tuning the behavior of text placed under bars. We use the so-called "new algorithm" which
%% places the bar in the middle of surrounding notes, and the text in the middle of surrounding text.
%% however, we restrict drastically the deviation of the text from the position of the bar.
\grechangedim{maxbaroffsettextleft}{0.5mm}{scalable}
\grechangedim{maxbaroffsettextright}{0.5mm}{scalable}

%% in case we show NABC, font selection
\gresetnabcfont{gregall}{12} 

%% \officepartannotation converts a letter (IHARPT) into the office part to be printed as annotation,
%% storing the result into \result.

\newcommand{\result}{}
\newcommand{\lookup}[3]{%
  \IfSubStr{#2}{#1}{ \renewcommand{\result}{#3} }{}%
}%
\newcommand{\officepartannotation}[1]{%
  \renewcommand{\result}{#1}%
  \lookup{#1}{T}{}%
  \lookup{#1}{H}{Hymn.}%
  \lookup{#1}{A}{Ant.}%
  \lookup{#1}{P}{}%
  \lookup{#1}{R}{Resp.}%
  \lookup{#1}{I}{Invit.}%
  \result%
}%

%% header capture setup for the mode
\newcommand{\defaultannotationshift}{-2mm}
\newcommand{\modeannotation}[1]{\greannotation{\hspace{\defaultannotationshift}#1}}
\gresetheadercapture{mode}{modeannotation}{string}

%% outputs a score with no label, indexing, initials or annotations
%% for 
\newcommand{\unindexedscore}[1]{
  \gresetinitiallines{0}
  %% the use of a directory called "gabc" is linked
  %% to the management of gabc files by the website: do not change 
  %% without also changing the website static files structure
  \gregorioscore{\subfix{nocturnale-romanum/gabc/#1}}
  \gresetinitiallines{1}
}

%% outputs a score with label, indexing, and annotations. no initials if [n] is passed
\makeatletter
\newcommand{\gscore}[5][y]{
  %% #1 (passed as option) : y = initial, n = no initial
  %% #2 : name of the score file, should be a code, e.g. Q4F4A3 or 1225N1R1
  %% #3 : office-part among the values: T, H, A, P, R, I (toni communes, hy., ant., psalmus, resp., invit.)
  %% #4 : if applicable, a number between 1 and 9 (rank of the ant./resp.) - else: empty
  %% #5 : the indexed name of the piece
  
  %% this prevents page breaks between the phantom section and its label, and the actual score.
  \needspace{2\baselineskip} 
  \protected@edef\@currentlabelname{#5}
  \phantomsection
  \label{#2}
  %% we add the office part, and number of that ant. or resp. in the current office, if applicable
  %% todo : the negative hspace is here because somehow the initial and annotation (first line only) are misaligned by 1mm _with this initial font size_.
  %% this should probably be fixed in a more elegant way.
  \greannotation[c]{\hspace{-1.4mm}\hspace{\defaultannotationshift}\officepartannotation{#3}#4}
  %% if #5 (indexed name) is blank, nothing is indexed.
  %% this is for pieces that are repetitions of another piece (antiphons after psalms)
  \ifblank{#5}{}{\index[#3]{#5}}
  %% if optional arg #1 has been passed as 'n', set no initial
  \ifx n#1\gresetinitiallines{0}\fi
  %% the use of a directory called "gabc" is linked
  %% to the management of gabc files by the website: do not change 
  %% without also changing the website static files structure
  \gregorioscore{\subfix{nocturnale-romanum/gabc/#2}}
  %% if optional arg #1 has been passed as 'n', unset no initial
  \ifx n#1\gresetinitiallines{1}\fi
  \vspace{1mm}
}
\makeatother

%%% We select if we want to show NABC. Here, we do.

\gresetnabc{1}{visible}
\gresetnabc{2}{invisible}

%%%%%%%%%%%%%%% FONTS %%%%%%%%%%%%%%%

%%%%%%%%%%%%%%% Main font
\setmainfont[Ligatures=TeX, Scale=\customscale]{Charis SIL}
%\setmainfont[Ligatures=TeX, Scale=\customscale]{TeXGyreBonum-Regular}
\setstretch{\fpevalc{1.05 * \customscale}}

%%%%%%%%%%%%%%% Score initials
%% \initialsize resizes the initials, with one argument (size in points)
\newcommand{\initialsize}[1]{
    \grechangestyle{initial}{\fontspec{Zallman Caps}\fontsize{#1}{#1}\selectfont}
}
%% default initial size is 32 points
\newcommand{\defaultinitialsize}{28}
\initialsize{\defaultinitialsize}

%% spacing before and after initials to kern the Zallman Caps.
%% this should be changed if we move away from Zallman Caps.
\newcommand{\initialspace}[2]{
  \grechangedim{afterinitialshift}{#2}{scalable}
  \grechangedim{beforeinitialshift}{#1}{scalable}
}
%% default space before and after initials is 0mm to the left and 2mm to the right.
\newcommand{\defaultinitialspace}{\initialspace{0mm}{-\defaultannotationshift}}
\defaultinitialspace{}

%%%%%%%%%%%%%%% Score annotations
\grechangestyle{annotation}{\small}

%%%%%%%%%%%%%%% GRAPHICAL ELEMENTS %%%%%%%%%%%%%%%

%% V/, R/, A/ and + signs for in-line use (\vv \rr \aa \cc)
\newcommand{\specialcharhsep}{3mm} % space after invoking R/ or V/ or A/ outside rubrics
\newcommand{\vv}{\textcolor{gregoriocolor}{\fontspec[Scale=\customscale]{Charis SIL}℣.\nolinebreak[4]\hspace{\specialcharhsep}\nolinebreak[4]}}
\newcommand{\rr}{\textcolor{gregoriocolor}{\fontspec[Scale=\customscale]{Charis SIL}℟.\nolinebreak[4]\hspace{\specialcharhsep}\nolinebreak[4]}}
\renewcommand{\aa}{\textcolor{gregoriocolor}{\fontspec[Scale=\customscale]{Charis SIL}\Abar.\nolinebreak[4]\hspace{\specialcharhsep}\nolinebreak[4]}}
\newcommand{\cc}{\textcolor{gregoriocolor}{\fontspec[Scale=\customscale]{FreeSerif}\symbol{"2720}~}}
%% Same special characters, for in-score use (<sp>V/ R/ A/ +</sp>)
\gresetspecial{V/}{\textcolor{gregoriocolor}{\fontspec[Scale=\customscale]{Charis SIL}℣.~}}
\gresetspecial{R/}{\textcolor{gregoriocolor}{\fontspec[Scale=\customscale]{Charis SIL}℟.~}}
\gresetspecial{A/}{\textcolor{gregoriocolor}{\fontspec[Scale=\customscale]{Charis SIL}\Abar.~}}
\gresetspecial{+}{{\fontspec[Scale=\customscale]{FreeSerif}†~}}
\gresetspecial{*}{\gresixstar}
\gresetspecial{cross}{\textcolor{gregoriocolor}{\fontspec[Scale=\customscale]{FreeSerif}\symbol{"2720}}}
\gresetspecial{labiacross}{\textcolor{gregoriocolor}{+}}
%% Same special characters, for use in rubrics (no space, and no red command since it will be reddified with the rest)
\newcommand{\vvrub}{{\fontspec[Scale=\customscale]{Charis SIL}℣.~}}
\newcommand{\rrrub}{{\fontspec[Scale=\customscale]{Charis SIL}℟.~}}
\newcommand{\aarub}{{\fontspec[Scale=\customscale]{Charis SIL}\Abar.~}}

%% the asterisk as found in the mediants of text-only psalms
\newcommand{\psstar}{\GreSpecial{*}}
\newcommand{\pscross}{\GreSpecial{+}}
%% also, most psalms do not call those but use † and * - todo

%% Roman Numerals
\usepackage{modroman}
\newcommand{\Rnum}[1]{\nbRoman{#1}}
\newcommand{\rnum}[1]{\nbshortroman{#1}}

%% Macro to print versicles
\newcommand{\versiculus}[2]{\par\vv #1 \\ \rr #2\par}

\newcommand{\versiculustpall}[2]
	{\versiculus{#1 \rubric{(T.P.} Allelúia.\rubric{)}}{#2 \rubric{(T.P.} Allelúia.\rubric{)}}}

%% Macro to print rubrics
\newcommand{\rubric}[1]{\textcolor{gregoriocolor}{\emph{#1}}}

%% Macro to print the name of a score in normal characters inside a \rubric
\newcommand{\normaltext}[1]{{\normalfont\normalcolor #1}}
\newcommand{\scorename}[1]{\normaltext{\nameref{M-#1}}}

%% Macro to print a full reference to a responsory
%% #1 is the R/ number in the feast, #2 is the R/ code, #3 is an optional additional text, like "sine Gloria Patri".
\newcommand{\respref}[3]{\rubric{%
\rrrub #1 \scorename{#2}, pag.\ \pageref{M-#2}%
\if\relax\detokenize{#3}\relax%
.%
\else%
, #3.%
\fi%
}\par}

\newcommand{\resprefcumgp}[2]
	{\respref{#1}{#2}{sed cum \normaltext{Glória Patri} in fine}}
	
\newcommand{\resprefsinegp}[2]
	{\respref{#1}{#2}{sine \normaltext{Glória Patri} in fine}}
	
\newcommand{\respnocturnref}[2]
	{\rubric{Tria Responsoria ut in \Rnum{#1} Nocturno Dominicæ præcedentis, pag.\ \pageref{M-#2}.}\par}

%% Macro to print the common rubric that signals the Te Deum
\newcommand{\tedeumrubric}{\rubric{Lectione ultima peracta Hymnus \normaltext{Te Deum} cantatur.}}

%% Macro to print the common rubric that signals the Paschaltide three-psalm one-antiphon rule
\newcommand{\tptresrubric}{\rubric{Tempore Paschali tres Psalmi sub hac Antiphona dicuntur.}}

%% Macros to print various rubrics related to commons
\newcommand{\respdetemp}{Responsoria de Tempore.}
\newcommand{\apexrubric}{Commune Apostolorum, pag.\ \pageref{M-APEX}.}
\newcommand{\aptprubric}{Commune Apostolorum Tempore Paschali, pag.\ \pageref{M-APTP}.}
\newcommand{\apvelrubric}{Commune Apostolorum, pag.\ \pageref{M-APEX} extra Tempus Paschale, aut \pageref{M-APTP} Tempore Paschali.}
\newcommand{\umexrubric}{Commune unius Martyris, pag.\ \pageref{M-UMEX}.}
\newcommand{\pmexrubric}{Commune plurimorum Martyrum, pag.\ \pageref{M-PMEX}.}
\newcommand{\umvelrubric}{Commune unius Martyris, pag.\ \pageref{M-UMEX} extra Tempus Paschale, aut \pageref{M-MRTP} Tempore Paschali.}
\newcommand{\mrtprubric}{Commune Martyrum Tempore Paschali, pag.\ \pageref{M-MRTP}.}
\newcommand{\coporubric}{Commune Confessoris Pontificis, pag.\ \pageref{M-COPO}.}
\newcommand{\conprubric}{Commune Confessoris non Pontificis, pag.\ \pageref{M-CONP}.}
\newcommand{\copodorubric}{Commune Confessoris Pontificis, pag.\ \pageref{M-COPO}, pro Doctore.}
\newcommand{\conpdorubric}{Commune Confessoris non Pontificis, pag.\ \pageref{M-CONP}, pro Doctore.}
\newcommand{\vnrubric}{Commune Virginum, pag.\ \pageref{M-MU}, pro non Martyre.}
\newcommand{\vmrubric}{Commune Virginum, pag.\ \pageref{M-MU}, pro Martyre.}
\newcommand{\nnrubric}{Commune non Virginum, pag.\ \pageref{M-MU}, pro non Martyre.}
\newcommand{\nmrubric}{Commune non Virginum, pag.\ \pageref{M-MU}, pro non Martyre.}
\newcommand{\cbmvrubric}{Commune Beatæ Mariæ Virgnis, pag.\ \pageref{M-CBMV}.}
\newcommand{\cdedrubric}{Commune Dedicationis Ecclesiæ, pag.\ \pageref{M-CDED}.}

%% Macro to print a separator
\newcommand{\sep}{{\centering\greseparator{3}{20}\par}}

%% Macro to insert vertical space, such that there be only six staves with NABC on a page, where typically there are seven.
\newcommand{\sixlinesvspace}{\vspace{14mm}}

%%%%%%%%%%%%%%% TABLE OF CONTENTS CONFIG %%%%%%%%

\setlength{\cftbeforetoctitleskip}{0cm}
\renewcommand{\cfttoctitlefont}{\hspace*{\fill}\Large\sc}
%% this, combined with the hspace above, is for centering the title. \centering does not work
\renewcommand{\cftaftertoctitle}{\hspace*{\fill}\vfill\thispagestyle{empty}}

%%%%%%%%%%%%%%% COLUMN MANAGEMENT %%%%%%%%%%%%%%%

\usepackage{multicol}
\usepackage{parcolumns}
\setlength\columnseprule{0.4pt}
\setlength{\multicolsep}{6pt plus 2pt minus 1.5pt}

%% Macro to print a psalm on two columns.
\newcommand{\psalmus}[2]{
	\vspace{0.5\baselineskip}
	\needspace{5\baselineskip}
	\phantomsection
	\label{Psalm#1_#2}
	\index[P]{#1 (mode #2)}
	\smalltitle{Psalmus #1}
	\begin{multicols}{2}
	\begin{itemize}[
		label=\null, 
		leftmargin=0pt, 
		itemindent=0pt, 
		labelsep=0pt, 
		labelwidth=0pt, 
		rightmargin=0pt, 
		parsep=0pt, 
		itemsep=0pt]
	\input{psalmi/#1_#2.tex}
	\end{itemize}
	\end{multicols}
}

\newcommand{\twocolrubric}[2]{
	\begin{parcolumns}[rulebetween]{2}
	\colchunk{%
      \rubric{#1}
	}
	\colchunk{%
	  \rubric{#2} 
    }
	\end{parcolumns}
}

%%%%%%%%%%%%%%% HEADER STYLES %%%%%%%%%%%%%%%

\pagestyle{fancy}
\fancyhead{}
\fancyfoot{}
\renewcommand{\headrulewidth}{0pt}
\setlength{\headheight}{20pt}
\fancyhead[RO]{\small\rightmark\hspace{1cm}\thepage}
\fancyhead[LE]{\small\thepage\hspace{1cm}\leftmark}

% this command is called every time the left and right header texts are set (e.g. by calling \feast)
% the hyphenpenalty override is neede only on older versions of gregorio which do not reset it correctly after typesetting the score.
% see https://tex.stackexchange.com/questions/581013/lualatex-hyphenation-issue-in-fancyhdr-with-gregoriotex-and-multicols-latin-te
\newcommand{\setheaders}[2]{
	\renewcommand{\rightmark}{\hyphenpenalty=50{\sc#2}}
	\renewcommand{\leftmark}{\hyphenpenalty=50{\sc#1}}
}
\setheaders{}{}

%%%%%%%%%%%%%%% TITLE STYLES %%%%%%%%%%%%%%%

%% Titles are centered and small-caps
\titleformat{\chapter}[block]{\Large\filcenter\sc}{}{}{}
\titleformat{\section}[block]{\large\filcenter\sc}{}{}{}
\titleformat{\subsection}[block]{\filcenter\sc}{}{}{}
\setcounter{secnumdepth}{0}
%% Fine-tuning of space around titles
\titlespacing*{\paragraph}{0pt}{1ex}{.6ex}

%% Typesets all titles throughout the NR except Nocturn titles and a few special titles.
%% Using a continuation is necessary because there are 11 arguments.
%% Only \feast, \nocturn, \intermediatetitle and \smalltitle should ever be used.
\newcommand{\feast}[6]{
  %% #1: feast code, e.g. 1225 or A1F1
  %% #2: feast title
  %% #3: left header title
  %% #4: right header title
  %% #5: title level
    %% title level 1 : full page width, a few major feasts + titles of temporale, sanctorale, etc.
	%% title level 2 : all feasts, sundays and major ferias
	%% title level 3 : ferias
  %% #6: incipit date (goes above feast title)
  %% cont'd #1: 1954 rank
  %% cont'd #2: 1960 rank
  %% cont'd #3: name of the feast as it shows up in the index
  %% cont'd #4: 1945 feast-wide rubrics
  %% cont'd #5: 1960 feast-wide rubrics

  %% needspace: should be barely more than the vertical space for the titles, rubrics excluded.
  %% this is to ensure that the page does not get cut after the title or the phantomsection.
  \needspace{8\baselineskip}
  %% phantomsection is to allow the label to attach to the title and not the previous counter object.
  \phantomsection
  \label{#1}
  \begin{center}
  %% we typeset a line for the date if the date is not blank
  \ifblank{#6}{}{
    {\large #6}\\%
  }%
  %% the actual title
  \ifx 1#5{\setstretch{1.2}\sc\huge #2\par}\fi
  \ifx 2#5{\Large #2\par}\fi
  \ifx 3#5{\large #2\par}\fi
  \end{center}
  %% If this is a level 1 title, empty pagestyle
  \ifx 1#5\thispagestyle{empty}\fi
  
  %% we define the header titles manually
  \setheaders{#3}{#4}
  
  %% moving on to a continuation macro to unpack the last 5 arguments
  \feastcontinued
}
\newcommand{\feastcontinued}[5]
{
  %% we name the last 5 arguments
  \def\oldrank{#1}%
  \def\newrank{#2}%
  \def\indexfeastname{#3}%
  \def\oldrubric{#4}%
  \def\newrubric{#5}%
  %% we index the feast if the indexing name is given
  \ifblank{#3}{}{
	\index[F]{\indexfeastname}
  }
  %% we typeset a two-column rank & rubrics block if one rank is filled in
  %% if the left rubric is filled in, we add a rule between the columns
  \ifblank{#5}{\def\ourrule{}}{\def\ourrule{rulebetween}}
  \ifblank{#1}{}{\vspace{-2mm}%
    \begin{parcolumns}[\ourrule]{2}
	\colchunk{%
      {\centering\oldrank\par}\rubric{\oldrubric}}%
	\colchunk{%
	  {\centering\newrank\par}\rubric{\newrubric}}%
	\end{parcolumns}
	\vspace{2mm}
  }
}

\newcommand{\smalltitle}[1]{
  \needspace{5\baselineskip}
  \par{\centering\textbf{#1}\par}
}

\newcommand{\intermediatetitle}[1]{
  \needspace{8\baselineskip}
  \begin{center}
  {\large #1}
  \end{center}
}

\newcommand{\nocturn}[1]{
  \intermediatetitle{In \Rnum{#1} Nocturno}
}

%% command to wrap printindex and set the headers for indices
\newcommand{\cprintindex}[2]{
	\setheaders{Indices}{#2}
	\pagestyle{fancy}
	\thispagestyle{empty}
	\printindex[#1]
}

%%%%%%%%%%%%%%% SUBFILES %%%%%%%%%%%%%%%

\usepackage{xr}
\usepackage{subfiles}

%% When we start a new subfile (new chapter), 
%% we start on a new page (with blank filling on the previous page) and create a corresponding label.
\newcommand{\customsubfile}[1]{\newpage\label{#1}\thispagestyle{empty}\subfile{#1}}
