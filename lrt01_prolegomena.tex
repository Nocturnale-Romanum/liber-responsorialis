% !TEX TS-program = lualatex
% !TEX encoding = UTF-8

\documentclass[liber-responsorialis_temporale.tex]{subfiles}

\ifcsname preamble@file\endcsname
  \setcounter{page}{\getpagerefnumber{M-lrt01_prolegomena}}
\fi

\begin{document}

%%rewrote some sentences to make them more English even where I could see exactly how it worked in French in my head (not even looking at the French

\begin{titlepage}
\begin{center}

\null\vspace{5mm}
{\Large\sc{}Nocturnale Romanum}

\vspace{5mm}

{\large\sc{}Tomus VIa}

\vspace{3cm}

{\Huge{}LIBER RESPONSORIALIS}

\vspace{1cm}

{\Large\sc{}pro nocturnis horis\\in omnibus diebus\\per totum anni circulum}

\vspace{5mm}

{\large\sc{}secundum ordinem Divini Officii\\a Pio pp X restituti\\cum variationibus a Joanne pp XXIII instauratis}

\vspace{5mm}

{\Large\sc{}DE TEMPORE}

\vspace{3.5cm}

{\large\sc{}instrumentum laboris}

\vfill

MMXXIV

\end{center}
\end{titlepage}

\null

\feast{OR}{Proœmium}
	{Proœmium}{Proœmium}{2}{}{}{}{}{}{}
\thispagestyle{empty}
\addcontentsline{toc}{section}{Proœmium}

\begin{paracol}[1]*{2}
{\grechangestyle{nabc}{\color{gregoriocolor}}

\begin{english}
	
\intermediatetitle{An adventure of silence}

{\setlength{\parindent}{5mm}\small

The night hour of the Divine Office
holds unforeseen treasures of prayer and life, as vast as they are unspeakable: one needs to simply sing it.

After the publication of the \emph{Invitatoriale} and the \emph{Antiphonale} for major feasts, we hope that this book will enable the discovery of these treasures, which grow as they are shared, for as many as possible.

This book is also, and most importantly at present, a \textbf{draft}. We, the editing team, are entirely too few to bring it up to proper standards by ourselves. 
Please give your feedback at\\{\footnotesize\url{https://github.com/Nocturnale-Romanum/nocturnale-romanum/issues}}

}

\end{english}

\switchcolumn

\intermediatetitle{Une aventure de silence}

{\setlength{\parindent}{5mm}\small

L'heure nocturne de l'Office divin recèle des trésors insoupçonnés de prière et de vie, aussi grands qu'ils sont peu communicables: il faut simplement la chanter.

Après la publication de l'invitatorial et de l'antiphonaire festif, nous espérons que cet ouvrage permettra au plus grand nombre de découvrir ces trésors, qui croissent d'autant qu'ils sont partagés.

Ce livre est aussi, et surtout --- pour le moment --- un \textbf{brouillon}. L'équipe d'édition est beaucoup trop réduite pour en amener la qualité, par elle-même, à un niveau satisfaisant.
Merci de bien vouloir signaler les corrections nécessaires sur\\{\footnotesize\url{https://github.com/Nocturnale-Romanum/nocturnale-romanum/issues}}

}

\switchcolumn*

\begin{english}

\intermediatetitle{On the rhythm of Gregorian Chant\\as considered in this edition}

{\setlength{\parindent}{5mm}\small

This \emph{Liber Responsorialis} is intended for all those who wish to sing the night hour of Divine Office, or at least access its responsories easily and confidently, whatever their approach to rhythm.

To this end, the editors wish to briefly expose the principles that have informed the correspondence between neumatic notation and horizontal \emph{episemata} in this book.

In Gregorian Chant, notes have a base length, 
henceforth the \emph{syllabic value}. 
It is the value of a syllable sung on a single note, when cantillation has the text sung on one note for each syllable.

When the neume for a syllable sung on a single note receives an \emph{episema}, or a letter T (\emph{tenete}), 
this note takes on the \emph{long value}, that is longer than the syllabic value. 
In this case, the square note receives an episema as well in this edition.

Notes within a melisma (that is, a syllable sung on more than one note) receive by default the \emph{short value}, 
a value that is shorter than the syllabic value. 
Frequently, their neumes receive an \emph{episema}, or an angled shape distinct from the usual one, or a neumatic break where the complex neume disintegrates into several simpler shapes, 
in which case the notes receive the syllabic value.
In this case, the square notes receive an episema as well, except notes before a \emph{quilisma}: 
it is common knowledge that those are to be somewhat lengthened even if they are not marked with an \emph{episema}.

{\gresetnabc{1}{visible}
\gresetclef{invisible}
\gresetinitiallines{0}
\begin{center}
\begin{minipage}[c]{0.7\textwidth}
\gregorioscore{\subfix{nocturnale-romanum/gabc/rhythmica_en}}
\end{minipage}
\end{center}
}

~

The proportion between the \emph{short value} and the \emph{syllabic value}, 
and between the \emph{syllabic value} and the \emph{long value}, should be consistent, but is at the cantor's discretion. 

It should also take into account the nature of the note: a \emph{stropha} marked with an \emph{episema} receives a 
syllabic value somewhat shorter than that of a \emph{virga} also marked with an \emph{episema}, 
the \emph{stropha} itself being slightly shorter than the \emph{virga}.

Finally, the end of a musical or textual phrase naturally brings about a certain lengthening of the notes all while the sound diminishes. 
This is customarily indicated by the \emph{punctum mora} or \emph{mora} dot.

}

%%changed sentence to phrase; translations of Mocquereu use this for the Fr. phrase, but it's also applicable to what he calls the incise and member
%% (which is so obvious that it's not always specified as being the case, just like in modern music or whatever)
%%added the missing part about volume from the Fr. ;)

\end{english}

\switchcolumn

\intermediatetitle{Le rythme du chant grégorien\\tel qu'il a servi à préparer cette édition}

{\setlength{\parindent}{5mm}\small

Ce \emph{Liber Responsorialis} s'adresse à toutes les personnes qui désirent chanter l'heure nocturne de l'Office divin, 
ou du moins avoir un accès facile et sûr à ses répons --- quelle que soit leur approche du rythme.

Pour cela, les éditeurs souhaitent exposer brièvement les principes qui, dans cet ouvrage, 
ont informé la correspondance entre la notation neumatique antique et l'usage désormais bien connu des épisèmes horizontaux.   

Dans le chant grégorien, les notes ont une longueur de base, qu’on peut appeler \emph{valeur syllabique}. 
Celle-ci est la valeur d’une syllabe chantée sur une seule note --- lorsque le texte de la mélodie est cantillé, une note sur chaque syllabe. 

Quand le signe neumatique associé à une syllabe chantée sur une seule note, a reçu dans les manuscrits un épisème, ou la lettre T (\emph{tenete}), 
cette note a une valeur longue, plus longue que la valeur syllabique. 
Dans notre publication, un épisème horizontal est ajouté à la note carrée.

Au sein d’un mélisme, c’est-à-dire au sein d’un enchaînement de plusieurs notes sur la même syllabe, 
les notes ont une valeur plus courte que la valeur syllabique. 
Souvent, les signes neumatiques qui les transcrivent, ont reçu dans les manuscrits un épisème, 
ou bien leur forme a été modifiée par rapport à l’usage habituel, ou encore on constate une «coupure» neumatique; 
dans ces cas, les notes concernées ont une valeur syllabique, 
et notre publication ajoute des épisèmes horizontaux aux notes carrées concernées, 
sauf pour les notes qui précèdent un quilisma, dont on sait qu’elles sont toujours légèrement allongées.

{\gresetnabc{1}{visible}
\gresetclef{invisible}
\gresetinitiallines{0}
\begin{center}
\begin{minipage}[c]{0.8\textwidth}
\gregorioscore{\subfix{nocturnale-romanum/gabc/rhythmica_fr}}
\end{minipage}
\end{center}
}

~

Les rapports entre la \emph{valeur courte} et la \emph{valeur syllabique}, 
et entre la \emph{valeur syllabique} et la \emph{valeur longue}, doivent être cohérents entre eux, mais sont à la main du chantre en fonction de l'acoustique du lieu.

La nature des notes devrait être aussi prise en compte: 
une \emph{stropha} épisémée prend une valeur syllabique plus légère que celle d'une \emph{virga} épisémée, 
la \emph{stropha} étant par nature plus légère que la \emph{virga}.

Enfin, le chant s'élargit naturellement, tout en diminuant de volume, à la fin d'une phrase textuelle ou musicale. C'est ce qui est indiqué, selon la coutume, par le point \emph{mora}.

}

\switchcolumn*

\begin{english}

\intermediatetitle{On clef changes and scale changes}

{\setlength{\parindent}{5mm}\small

Sometimes the editors of this book have needed to change the clef for purely typographical reasons, 
because the melody extended too much above or below the staff. 
In such a case, the note indicated by the \emph{custos} before the clef change, and the first note after, 
have the same height and the same name (solmization).


\begin{center}
\begin{minipage}[c]{0.9\textwidth}

{\gresetinitiallines{0}
\gabcsnippet{

(c4) Ordinary clef change(fh/gh/ggof.//f+//;c3//d//f//h) (::///////////)

(c4) Clef change with scale change(fh/gh/ggof.//f+//;c3//e//g//i) (::)

}
}

\end{minipage}
\end{center}


~

In some rare cases, this book features changes of scale, where, from one point forward, the piece uses a different
scale than at its beginning, that is, a different set of notes.
In such a case, the note indicated by the \emph{custos} before the clef change, and the first note after, 
are sung at the same height, but have different names, one belonging to the scale before the change, 
and one to the scale after the change.

Scale changes can be difficult to sing, and while the editors of this book believe that they reflect
the musical truth of the responsory in question, performing or ignoring them is left to the cantor's discretion.

}

%%it is true that "…discretion is advised" is an English expression but it feels wrong here.

\end{english}

\switchcolumn

\intermediatetitle{Changements de clef et changements d'échelle}

{\setlength{\parindent}{5mm}\small

Les éditeurs de cet ouvrage ont parfois introduit un changement de clef pour des raisons purement typographiques,
pour éviter une mélodie trop au-dessus ou en-dessous de la portée.
Dans un tel cas, la note indiquée par le guidon avant le changement de clef, et la première note qui le suit,
ont la même hauteur et le même nom.

\begin{center}
\begin{minipage}[c]{0.9\textwidth}

{\gresetinitiallines{0}
\gabcsnippet{

(c4) Changement de clef(fh/gh/ggof.//f+//;c3//d//f//h) (::///////////)

(c4) Changement d'échelle(fh/gh/ggof.//f+//;c3//e//g//i) (::)

}
}

\end{minipage}
\end{center}

~

Dans certains cas rares, cet ouvrage contient des changements d'échelle, où la pièce continue
en employant une échelle différente de celle avec laquelle elle a commencé --- c'est à dire un autre ensemble de notes.
Dans un cas pareil, la note indiquée par le guidon avant le changement de clef, et la première note qui le suit,
ont la même hauteur, mais pas le même nom, selon l'échelle où on considère cette note.

Les changements d'échelle sont d'une exécution délicate. Les éditeurs de cet ouvrage croient que ces 
changements reflètent la vérité musicale de la pièce, mais leur exécution ou non est laissée à la discrétion du chantre.

}

\switchcolumn*

\begin{english}

\intermediatetitle{On the notation of accidentals}

{\setlength{\parindent}{5mm}\small

Flat, sharp and natural signs are used in this book. 

Flats and sharps apply until the end of the word, or until the next bar line in a long melisma. 
In particular, their effect may carry over the end of the staff into the next.

Some pieces have a flat as a key signature. In this case, what has been said above regarding the flat and sharp signs is also true of the natural sign.

This sign also makes explicit the end of the effect of an accidental before the end of the word or the next bar line in the melisma; the natural sign is used in many cases in this edition.
}

\end{english}

%%had to rewrite this; "intervenes" does work as a verb of course but "intervening" as an adj. feels better, yet getting the two to work was a bit tricky.

\switchcolumn

\intermediatetitle{Notation des altérations}

{\setlength{\parindent}{5mm}\small

Les signes dièse, bémol et bécarre sont employés dans cet ouvrage.

Les signes dièse et bémol sont applicables jusqu'à la fin du mot, ou jusqu'à la prochaine barre, s'il s'agit d'un long mélisme.
En particulier, leur effet peut se prolonger d'une portée dans la suivante, au sein du même mot et si aucune barre n'est présente.

Certaines pièces ont un bémol à la clef. Dans ce cas, ce qui a été dit ci-dessus du dièse et du bémol s'applique au signe bécarre.

Cependant, dans de très nombreux cas, la fin de l'effet d'une altération a été rendue explicite par l'ajout de l'altération contraire.
}

\switchcolumn*

\begin{english}

\intermediatetitle{On the rubrics}

{\setlength{\parindent}{5mm}\small

Rubrics are typeset on two columns.
On the left column are the rubrics of the Roman Office
as it stood in 1954 before the reforms of Pius XII.
On the right column are the rubrics of the Roman Office
as it stood in 1962 after the reforms of John XXIII.

The rubrics given are not exhaustive, but rather reminders: 
it is assumed that the master of ceremonies has access to an \emph{Ordo}.

According to the 1960 rubrics, on Sundays of Advent, Septuagesima and Lent, the first, third and ninth responsories are used, omitting the \emph{Gloria Patri} of the third responsory. 
On other Sundays, the first and third responsories are used and the \emph{Te Deum} is sung after the third lesson.

}

%% The date of 1969 is not employed because Tres abhinc annos changed the rubrics again, so we are not compliant with 1969.

\end{english}

\switchcolumn

\intermediatetitle{Au sujet des rubriques}

{\setlength{\parindent}{5mm}\small

Les rubriques sont données sur deux colonnes. 
En colonne de gauche, les rubriques de l'Office romain
tel qu'il était en 1954 avant les réformes de Pie XII.
En colonne de droite, les rubriques de l'Office romain
tel qu'il était en 1962 après les réformes de Jean XXIII.

Les rubriques données ici ne sont pas exhaustives, mais plutôt des rappels: on suppose que le cérémoniaire a accès à un \emph{Ordo}.

Selon les rubriques de 1960, aux dimanches de l'Avent, de la Septuagésime et du Carême, les premier, troisième et neuvième répons sont employés, en omettant
le \emph{Gloria Patri} du troisième répons.
Aux autres dimanches, le premier et le troisième répons sont employés, et le \emph{Te Deum} est chanté après la troisième leçon.

}

\switchcolumn*


\begin{english}

\intermediatetitle{On critical restitution}

{\setlength{\parindent}{5mm}\small

This book is decidedly a practical edition. It attempts to be critically informed in the following ways.

The neumes indicated above the staff (in versions of this book that have them) are from the Hartker Antiphonary, and in some very rare cases from other Antiphonaries using St.\ Gall notation with rhythmic indications. Responsories and Verses absent from this manuscript have been given synthetic neumes between brackets.

In some rare cases, when the manuscripts agree on a text that is not markedly different from that found in the Roman Breviary, this text was used instead of that found in the  Breviary, as was done for the day hours in the 1912 Roman Antiphonary.

}

\end{english}

\switchcolumn

\intermediatetitle{Restitution et démarche critique}

{\setlength{\parindent}{5mm}\small

Cet ouvrage est fondamentalement une édition pratique. Il adopte une démarche critique de deux manières:

Les neumes imprimés au-dessus de la portée (s'ils sont présents) sont ceux de l'antiphonaire de Hartker, et dans quelques cas très rares, ceux d'autres antiphonaires sangalliens avec notation rythmique. Les répons et versets absents de ce manuscrit ont reçu des neumes synthétiques entre crochets.

Dans quelques cas, quand les manuscrits donnent tous un texte qui n'est que légèrement différent du texte du bréviaire romain, ce texte a été conservé au lieu de celui du bréviaire, comme ce fut le cas pour les heures diurnes dans l'antiphonaire romain de 1912.

}

\switchcolumn*

\vspace{\baselineskip}
\sep

\switchcolumn

\vspace{\baselineskip}
\sep

\switchcolumn*

\vfill

\intermediatetitle{Tabella neumatum}

{\gresetnabc{1}{visible}
\gresetclef{invisible}
\gresetinitiallines{0}
\gregorioscore{\subfix{nocturnale-romanum/gabc/neumata}}
}

\vfill

\pagebreak

\switchcolumn

\vfill

{\footnotesize

\intermediatetitle{Editores}

\begin{center}
Matthias Bry

Dominique Crochu
\end{center}

\intermediatetitle{Socii}

\begin{center}
John Anderson

Dom Giacomo Frigo

Dominique Gatté

Dom Jacques-Marie Guilmard

Rob Leduc
\end{center}

\intermediatetitle{Amici}

\begin{multicols}{3}

\begin{center}
D. Guillaume Antoine

Olivier Berten

Charlotte Bry

Marie-Françoise Crochu

Dom Damien

Alberto Diaz-Blanco

Gregory DiPippo

Giedrius Gapsys

Joshua Guenther

Anne Guyard

Andrew Hinkley

Jakub Jelinek

Marek Klein

Joanna Klimowicz

Naïl Lazrak

Jakub Pavlik

R. P. Raphaël \textsc{crmd}

Matthew Roth

Fr. Simon-Marie \textsc{fsvf}

Dom Samuel Springuel

D. Henri Vallançon
\end{center}

\end{multicols}

\intermediatetitle{In Memoriam}

\begin{center}
Holger Peter Sandhofe

Louis-Marie Vigne

André Butet
\end{center}

}

\vfill

}%% end scope NABC color = gregoriocolor
\end{paracol}

\end{document}
